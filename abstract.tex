Developments in the field of AI in general, and Large Language Models (LLMs) in particular, have created a `perfect storm’ for observing `sparks’ of Artificial General Intelligence (AGI) that are spurious. Like simpler models, LLMs distill meaningful representations in their latent embeddings that have been shown to correlate with external variables. Nonetheless, the correlation of such representations has often been linked to human-like intelligence in the latter but not the former. We probe models of varying complexity including random projections, matrix decompositions, deep autoencoders and transformers: all of them successfully distill information that can be used to predict latent or external variables and yet none of them have ever been linked to AGI. We, therefore, argue that patterns in latent spaces are spurious sparks of AGI. Additionally, we review literature from the social sciences that shows that humans are prone to seek patterns and anthropomorphize. We conclude that both the methodological setup and common public image of AI are ideal for the misinterpretation that correlations between model representations and some variables of interest are `caused' by the model's understanding of underlying `ground truth’ relationships. We, therefore, call for the academic community to exercise extra caution, and to be keenly aware of principles of academic integrity, in interpreting and communicating about AI research outcomes.